% Esta template foi adaptado retirado de: https://www.overleaf.com/18680340cssjcshywjxx#/70316020/
% Obrigado pelo material

\documentclass[answers]{exam}

%% Language and font encodings
\usepackage[english]{babel}
\usepackage[utf8x]{inputenc}
\usepackage[T1]{fontenc}

%% Sets page size and margins
\usepackage[a4paper,margin=2cm]{geometry}

%% Useful packages
\usepackage{amsmath}
\usepackage{graphicx}
\usepackage{paralist}
\setlength\FrameSep{4pt}

\newcommand{\answer}[1] {
	\begin{framed}
		#1	
	\end{framed}	
}

\newcommand{\questao}[2]{
	\question[1]{#1}
	\answer{#2}
}

\begin{document}
	\begin{questions}
		\questao{Discorra sobre a Era da informação}{
			Atualmente vivemos em uma era onde os dados são utilizados para praticamente tudo, desde propagandas mais acertivas, até formas de controle produtivo de fábricas e lavouras. Nesta era a tecnologia apresenta um papel fundamental em prover facilidades, sendo ela a principál responsável pelas grandes mudanças no mundo atual.	
		}

		\questao{Quais os elementos e protocolos que direcionam uma comunicação humana bem sucedida ?}{
			São eles:
			\begin{itemize}
				\item Um emissor e um receptor identificados;
				\item Acordo sobre o método de comunicação (Telefone, carta, etc);
				\item Língua comum;
				\item Velocidade e ritmo de transmissão;
				\item Requisitos de confirmação ou recepção.
			\end{itemize}
		}		
		
		\questao{Que fatores externos afetam a comunicação em rede ?}{
			Os fatores externos são aqueles relacionados a complexidade da rede e ao número de dispositivos presentes nela, os principais fatores externos que afetam a comunicação em uma rede podem ser vistos abaixo:
			\begin{itemize}
				\item A qualidade do caminho entre emissor e receptor;
				\item O número de vezes que uma mensagem tem que mudar de forma;
				\item O número de vezes que uma mensagem tem que ser redirecionada ou reenviada;
				\item A quantidade de mensagens transmitidas simultaneamente na rede de comunicação;
				\item O tempo designado para uma comunicação bem sucedida.
			\end{itemize}		
		}	
			
		\questao{Que fatores internos interferem na comunicação de rede ?}{
			Fatores internos são aqueles relacionados a própria natureza da mensagem, isto ocorre já que, as mensagens podem ter diferentes tipos de complexidade e importância, fezendo assim com que haja diferenciação entre mensagens no momento da transmissão na rede.
			Dentre estes fatores, inclui-se:
			\begin{itemize}
				\item O tamanho da mensagem;
					\begin{itemize}
						\item Mensagens muito grandes podem ser interrompidas ou atrasadas em determinados pontos da rede.
					\end{itemize}
				\item A complexidade da mensagem;
				\item A importância da mensagem.
					\begin{itemize}
						\item Em casos de redes sobrecarregadas, pacotes com baixa prioridade são descartados
					\end{itemize}
			\end{itemize}
		}
				
		\questao{Quais os quatro elementos básicos que todas as redes possuem ?}{
			São eles:
			\begin{itemize}
				\item Regras;
					\begin{itemize}
						\item Estes determinam como as mensagens são enviadas/direcionadas/recebidas/interpretadas.
					\end{itemize}
				\item Meio físico;
					\begin{itemize}
						\item São os meios físicos que fazem a interligação dos dispositivos em uma rede, sendo estes os responsáveis em transportar as mensagens enviadas de um dispositivo para outro.
					\end{itemize}
				\item Mensagens;
					\begin{itemize}
						\item Informação/Conteúdo que navega entre os dispositivos.
					\end{itemize}
				\item Dispositivos.
			\end{itemize}					
		}
		
		\questao{O que é convergência de redes ?}{
			Convergência de rede pode ser entendido como uma única plataforma de rede que possibilita a utilização de diversos serviços sem a distinção destes, assim, não sendo necessário um equipamento específico para cada tipo de serviço que funcionada sobre a rede. Esta característica das redes nem sempre foi assim, no início cada serviço tinha sua rede, assim telefone tinha uma rede diferente das redes de computadores, e hoje graças a convergência não existe mais esta diferenciação, pode-se ter serviços de telefone na mesma rede de computadores.
			Isto ocorre pois, tudo começou a ser tratados como dados, estes que por sua vez não tem destinção do que carrega.
		}
				
		\questao{Defina rede LAN}{
			Este é um tipo de rede que se espalha por uma única área geográfica, fornecendo serviços e aplicações a pessoas de uma estrutura organizacional comum, como por exemplo para um negóco ou organização.
			Nestas redes há uma pequena região onde os dispositivos estão conectados, normalmente esta pequena região pode ser entendida como um escritório ou uma residência.		
		}
					
		\questao{O que é Rede de Computador ?}{
			De acordo com Tannenbaum, "Coleção de computadores autônomos interconectados por uma só tecnologia. Entende-se que computadores conectados são aqueles capazes de trocar informação entre eles."		
		}
		
		\questao{Defina uma WAN}{
			São redes que abrangem grandes regiões geográficas, normalmente oferecidos por ISPs (\textit{Internet Service Provider}). Estas são redes utilizadas para interligar redes LANs de direntes locais, com diferentes distâncias.		
		}
		
		\questao{O que é a internet ?}{
			São redes interconectadas, onde há interconexões de diferentes redes, possibilitando a comunicação global entre as redes. Estas redes interconectadas pertencem a provedores de internet (ISPs).	
		}
		
		\questao{O que é Intranet ?}{
			Intranet é um termo normalmente usado para se referir a uma conexão privada de LANs e WANs que pertencem a uma organização. Esta normalmente é criada para ser acessível somente pelos membros da organização, funcionários ou outros com autorização.		
		}
		
	\end{questions}
\end{document}
